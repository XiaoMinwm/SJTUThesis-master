%# -*- coding: utf-8-unix -*-
%%==================================================
%% conclusion.tex for SJTUThesis
%% Encoding: UTF-8
%%==================================================



\chapter{总结与展望}    

\section{工作总结} 

PCS的网络物理安全问题变得越来越严重,从网络信息安全和物理数据完整性的角度来看,PCS的安全风险和入侵威胁正在增加。事实上大多数控制应用(例如监控和数据采集(SCADA))都需要保证严格的安全性,因为一旦由任何异常故障或恶意攻击引起的错误甚至恶意行为都会导致物理设备甚至整个系统遭受无法逆转的破害。因此,PCS网络信息安全和数据安全已经越来越需要全面深入地研究国家基础设施和关键领域。为了检测PCS的入侵威胁并建立安全系统和关键的基础设施安全的综合防御系统,本文针对PCS网络物理安全的入侵类型,控制器本身及其与物理设备数据漏洞特征,面向PCS的攻击和入侵检测方法的研究与设计主要工作有:
\begin{enumerate}
\item 基于故障检测机制构造故障序列注入(FSI)攻击。它可以避免现有的系统故障检测机制,感染和篡改PLC的输入信号,迫使PLC进行误操作,破坏控制系统的关键设备。首先,我们收集在PLC控制器和物理设备之间交换的信号或堆栈数据。然后我们使用输入和输出向量数据库来识别类似于故障检测建模方法的无故障离散事件模型。最后,搜索故障机制无法检测到的所有假序列,获得适当长度的恶意序列,对受控传感器收集的输入信号进行攻击。值得注意的是对多个分支状态的检测将成为对这种攻击的有效防御。实验仿真表明我们的方法是对部署故障检测的控制系统造成一定的破坏性威胁并证明我们提出的方法的有效性。

\item 提出了基于异常数据的入侵检测和定位防御机制,能够应对包括传统蛮力和类故障攻击在内的新型隐蔽的错误序列注入攻击从而更好地保护过程控制系统。我们给出了基于异常入侵检测的整个实现过程,包括对错误序列注入攻击的构造并注入系统控制器的输入信号中。针对异常数据注入攻击的特点,分为两个阶段设计异常数据检测算法。第一阶段将监测的控制器输入信号与模型的预期输出残差对比判断是否数据异常;第二阶段对隐蔽的错误序列注入攻击设计额外FSI检测算法来避免此类型的攻击;最后对检测存在异常的数据分析判断,定位到具体攻击源并给出相应的应对措施。最后基于Dspace的实验仿真表明,我们的检测能够对面临异常数据注入的攻击威胁起到很好的保护作用并证明我们提出的方法的有效性。

\item 基于规范的入侵检测机制,其被设计为保护可编程控制器本身的程序和指令免受恶意代码注入。只有经过验证的程序和指令才能操作系统或控制服务器上传到指定的可编程控制器设备。我们给出了基于规范入侵检测的整个实现过程,包括对可编程控制器程序进行形式化建模。考虑到PLC程序的存在定时器和计数器等高级语言没有的指令,我们通过抽象建模给出等价的可执行二进制逻辑函数对其进行解释建模,这极大的增强了我们建模算法的通用性而不仅仅局限于基于运算指令。最后基于自动重合闸控制系统的实验仿真表明,我们的检测能够对控制系统面临恶意代码注入的攻击威胁起到很好的保护作用并证明我们提出的方法的有效性。
\end{enumerate}

\section{工作展望} 

本文重点介绍了针对过程控制系统控制器层面的入侵检测系统的研究与设计,但总体而言,控制器的入侵检测研究还处于起步阶段,特别是在国内相对研究较少甚至无人问津。因此,针对底层控制器的入侵检测系统的研究仍需要大量的研究工作。在本文中,以下问题值得进一步研究:
\begin{enumerate}
\item 过程控制系统内各层次入侵检​​测协作和优化,从而提高入侵检测的整体效率和性能。包括共享入侵信息和检测信息,对建模和验证所需的计算资源优化调度等合作与优化机制。针对这个问题,本文提出了第三章基于异常数据检测和第四章基于安全规范检测两个层面的检测协调为一体,共同对控制器对象可能存在的内外攻击进行防护。然而,由于入侵检测方法的设计集中在现场设备层和过程监控层,不太容易进行深入研究和实现。而且对于不同的攻击对象,想建立一个通用的模型难度较大,所以协作分层设计检测机制是我们下一步重点研究的内容。
\item 在本文中,入侵检测方法对于现场设备层尤其针对可编程控制器设计,仅作为PLC控制器作为对象进行设计和测试验证,而没有更广泛的考虑其他现场控制器,比如常用的DSP和RTU。对于这个问题,进一步的工作可以为其他现场控制器设计入侵检测方法,也可以设计用于各种现场控制器更为通用的入侵检测方法。
\item 对于过程控制系统的入侵检测方法,检测精度和实时要求高于传统IT系统,但检测精度和实时性往往存在一定的负相关性。实时在线检测势必需要计算机更快的运算速度和时间,这会严重影响到检测准确度和精度。反之更高的检测精度也会拖累在线检测的实时性。所以进一步的研究可以设计具有更高的检测精度,更好的实时入侵检测方法,并且研究在线实时检测效率和检测精度的平衡和优化。
\end{enumerate}



